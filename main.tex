\input{preamble.tex}
\usepackage{graphicx} % support the \includegraphics command and options
\usepackage{hyperref}
\usepackage{amsmath}
\usepackage{float} 
\usepackage{listings} %so you can write computer code in the LaTex document. 
\begin{document} 
\title{Reducing Audio Bandwidth with an FFT-based Low-Pass Filter}

\author{
    \IEEEauthorblockN{Trent Geisler}
    \IEEEauthorblockA{Analytics and Data Science\\
    Kennesaw State University\\
    Kennesaw, Georgia 30144\\
    }
    \and
    \IEEEauthorblockN{Andrew Henshaw}
    \IEEEauthorblockA{Analytics and Data Science\\
    Kennesaw State University\\
    Kennesaw, Georgia 30144\\
    }
    \and
    \IEEEauthorblockN{Lauren Staples}
    \IEEEauthorblockA{Analytics and Data Science\\
    Kennesaw State University\\
    Kennesaw, Georgia 30144\\
    }
}

\maketitle

\begin{abstract}

    This project demonstrates the application of the Fast Fourier
    Transform (FFT) as a method for low-pass filtering of audio
    signals. We discuss the need for low-pass filtering to
    prevent undersampling and to reduce bandwidth requirements.
    We explain filter-design techniques and introduce GNU Radio
    as a platform for audio and radio-frequency exploration.  We
    explain and demonstrate how an appropriately-chosen low-pass filter
    can reduce bandwidth without noticeably degrading audio signal quality.

\end{abstract}

\IEEEpeerreviewmaketitle

\section{Introduction} 

Filtering an audio signal is extremely important for many
reasons. When an appropriate filter is designed and used to
remove any unwanted frequencies, then the bandwidth and power
necessary for audio signal transmission is reduced. Why is this
important? The frequency bands used for the transmission of many
types of signals are scarce resources. Every transmitter that can
interfere with others has to operate in a licensed band and is
subject to bandwidth limitations. This is why multiple radio
stations can operate in the same geographical area. They are
assigned different center frequencies and they have a certain
bandwidth they can occupy. If one radio station transmits beyond
their assigned bandwidth, they will impact the signal
transmission of other local radio stations. One way to ensure
that a radio station stays within their assigned bandwidth is
through the use of filtering.\cite{notes:class}  

Additionally, if you travel to a different city, you will find
radio stations that operate at the same frequency as the radio
stations located locally here in Atlanta. This is possible
because radio transmitters are limited in power. With unlimited
power, for example, one could hear FM 106.7 throughout the United
States and no other radio station would be able to use the
frequency 106.7 MHz. But power is limited, and high-power radio
transmissions cost a lot of money \cite{notes:class}. This is why
some radio stations like to brag about how powerful their
transmissions are. It is also why start-up radio stations have a
very limited range. Therefore, if power costs money, one does not
want to waste it by broadcasting frequencies that most humans
cannot hear.

Both bandwidth and power savings are the motivation behind Fast
Fourier Transform (FFT)-based low-pass filters. This paper will
demonstrate how a Finite Impulse Response (FIR) low pass filter
can remove unnecessary frequencies in order to reduce the
bandwidth and power required for transmission.  

\subsection{Objective}

Our objective is to design and demonstrate an FFT-based FIR
low-pass filter to reduce bandwidth of a recorded audio signal
for the purposes of AM Radio transmission.

\section{Methods}

We collected data, designed a low pass filter, pushed the signal
through the filter, and then evaluated the resulting signal.

\subsection{Data}

We collected an audio signal by recording the song
`Flight of the Bumblebee' at the full frequency spectrum that a compact disc (CD) is recorded, which is 44.1 kHz. The human ear does not even hear this full spectrum.  This means that there are potential bandwidth `savings'
for transmission! This collected audio recording is referred to
as the `raw signal.'

\subsection{Filter Design}

Beat Navy.  Test.

\subsection{Nyquist-Shannon Sampling Theorem}
Clemson beat Bama
\section{Evaluation of Filtered Signal}

We reduced the raw signal from 44.1 kHz to () kHz, 
with no perceived quality degradation to the three judges.

\section{Conclusion}
In radio operations, frequency bands are scarce resources that are governed by national and international agencies. Every transmitter must operate within its licensed band.  Both bandwidth and power limitations are the motivation for more efficient use of bandwidth. A common way of achieving more efficient use of bandwidth is to filter frequencies occurring above the hearing range of humans (20 kHz).  While humans cannot directly hear frequencies higher than this, these frequencies can still create tones through harmonics that are noticeable to humans.  However, often times these tones are subtle and do not necessarily warrant the expense of the bandwidth they require.  Our project designed several low-pass filters to find the point that optimizes the tradeoff in sound quality and bandwidth on a song called "Flight of the Bumble Bees."  Sound quality deterioration was not discernable until the 4 kHz filter and below.


\begin{thebibliography}{1}

\bibitem{GNU:radio}
\emph{GNU Radio}, \url{https://www.gnuradio.org/}.

\bibitem{notes:class} Baxley, R., Henshaw, A., Nowlan, S., \&
Trewhitt, E. \emph{Software-Defined Radio with GNU Radio: Theory
and Application, Georgia Tech Professional Education course
notes.} (2017)

\bibitem{human:rg} Rossing, Thomas (2007). Springer Handbook of
Acoustics. Springer. pp. 747, 748. ISBN 978-0387304465.

\bibitem{lyons:intro} Lyons, Richard G. Author. "Understanding
Digital Signal Processing." Ch. 1, 3, and 5. Web.

\bibitem{lowpass:wiki}
\url{https://en.wikipedia.org/wiki/Low-pass_filter}. Accessed
Nov. 7, 2019.

\bibitem{harris:rot} Harris, Fred J. "Multirate Signal Processing
for Communication Systems." Page 216, Equation 8.16.

\bibitem{aliase:wiki}
\url{https://en.wikipedia.org/wiki/Nyquist_frequency}. Accessed
Nov. 15, 2019.

\bibitem{notebook:sampling} Baxley, R., Henshaw, A., Nowlan, S.,
\& Trewhitt, E. \emph{Juyter Notebook for Sampling:
Sampling.ipynb.} (2017)

\bibitem{sinc:wiki}
\url{https://en.wikipedia.org/wiki/Nyquist_frequency}. Accessed
Nov. 16, 2019.

\bibitem{shannon:wiki}
\url{https://en.wikipedia.org/wiki/Nyquist\%E2\%80\%93Shannon_sampling_theorem}.
Accessed Nov. 20, 2019.

\bibitem{cd:wiki} \url{https://en.wikipedia.org/wiki/44,100_Hz}.
Accessed Nov. 20, 2019.

\end{thebibliography}

\newpage

\onecolumn 
\appendix 

\subsection{Python Code for Filter Design\cite{notes:class}}
\lstinputlisting[language = Python]{code/filter.py}

\newpage
\subsection{Demonstration Application}
\lstset{caption={Demonstration Application}}
\lstset{label={lst:demo}}
\lstinputlisting[language = Python]{code/bigdata.py}

% that's all folks
\end{document}


