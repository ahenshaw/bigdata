\input{preamble.tex}

\begin{document}
%
% paper title
% Titles are generally capitalized except for words such as a, an, and, as,
% at, but, by, for, in, nor, of, on, or, the, to and up, which are usually
% not capitalized unless they are the first or last word of the title.
% Linebreaks \\ can be used within to get better formatting as desired.
% Do not put math or special symbols in the title.
\title{Reducing Audio Bandwidth with an FFT-based Low-Pass Filter}

%Designing and demonstrating and FFT-based Low Pass Filter to reduce bandwidth of a recorded audio signal for the purposes of AM Radio transmission.


% author names and affiliations
% use a multiple column layout for up to three different
% affiliations
\author{\IEEEauthorblockN{Trent Geisler}
\IEEEauthorblockA{Analytics and Data Science\\
Kennesaw State University\\
Kennesaw, Georgia 30144\\
}
\and
\IEEEauthorblockN{Andrew Henshaw}
\IEEEauthorblockA{Analytics and Data Science\\
Kennesaw State University\\
Kennesaw, Georgia 30144\\
}
\and
\IEEEauthorblockN{Lauren Staples}
\IEEEauthorblockA{Analytics and Data Science\\
Kennesaw State University\\
Kennesaw, Georgia 30144\\
}}

% conference papers do not typically use \thanks and this command
% is locked out in conference mode. If really needed, such as for
% the acknowledgment of grants, issue a \IEEEoverridecommandlockouts
% after \documentclass

% for over three affiliations, or if they all won't fit within the width
% of the page, use this alternative format:
% 
%\author{\IEEEauthorblockN{Michael Shell\IEEEauthorrefmark{1},
%Homer Simpson\IEEEauthorrefmark{2},
%James Kirk\IEEEauthorrefmark{3}, 
%Montgomery Scott\IEEEauthorrefmark{3} and
%Eldon Tyrell\IEEEauthorrefmark{4}}
%\IEEEauthorblockA{\IEEEauthorrefmark{1}School of Electrical and Computer Engineering\\
%Georgia Institute of Technology,
%Atlanta, Georgia 30332--0250\\ Email: see http://www.michaelshell.org/contact.html}
%\IEEEauthorblockA{\IEEEauthorrefmark{2}Twentieth Century Fox, Springfield, USA\\
%Email: homer@thesimpsons.com}
%\IEEEauthorblockA{\IEEEauthorrefmark{3}Starfleet Academy, San Francisco, California 96678-2391\\
%Telephone: (800) 555--1212, Fax: (888) 555--1212}
%\IEEEauthorblockA{\IEEEauthorrefmark{4}Tyrell Inc., 123 Replicant Street, Los Angeles, California 90210--4321}}




% use for special paper notices
%\IEEEspecialpapernotice{(Invited Paper)}




% make the title area
\maketitle

% As a general rule, do not put math, special symbols or citations
% in the abstract
\begin{abstract}
The abstract goes here.
\end{abstract}

% no keywords




% For peer review papers, you can put extra information on the cover
% page as needed:
% \ifCLASSOPTIONpeerreview
% \begin{center} \bfseries EDICS Category: 3-BBND \end{center}
% \fi
%
% For peerreview papers, this IEEEtran command inserts a page break and
% creates the second title. It will be ignored for other modes.
\IEEEpeerreviewmaketitle



\section{Introduction}
The transmission of audio signals through radio frequency is bandwidth-limited, which is why the FCC regulates transmissions so closely. Keeping the frequency higher saves bandwidth, and the goal is to trim the frequencies without impacting the quality of what the human ear perceives. 

\subsection{Objective}
Our objective is to design and demonstrate an FFT-based low-pass filter to reduce bandwidth of a recorded audio signal for the purposes of AM Radio transmission.

\section{Methods}
We collected data, designed a low pass filter, pushed the signal through the filter, and then evaluated the resulting signal.

\subsection{Data}

We collected an audio signal by recording the song
`Flight of the Bumblebee' at the full frequency spectrum that a compact disc (CD) is recorded, which is 44.1 kHz. The human ear does not even hear this full spectrum.  This means that there are potential bandwidth `savings'
for transmission! This collected audio recording is referred to
as the `raw signal.'

\subsection{Filter Design}

Beat Navy.  Test.

\subsection{Nyquist-Shannon Sampling Theorem}
Clemson beat Bama
\section{Evaluation of Filtered Signal}

We reduced the raw signal from 44.1 kHz to () kHz, 
with no perceived quality degradation to the three judges.

\section{Conclusion}
In radio operations, frequency bands are scarce resources that are governed by national and international agencies. Every transmitter must operate within its licensed band.  Both bandwidth and power limitations are the motivation for more efficient use of bandwidth. A common way of achieving more efficient use of bandwidth is to filter frequencies occurring above the hearing range of humans (20 kHz).  While humans cannot directly hear frequencies higher than this, these frequencies can still create tones through harmonics that are noticeable to humans.  However, often times these tones are subtle and do not necessarily warrant the expense of the bandwidth they require.  Our project designed several low-pass filters to find the point that optimizes the tradeoff in sound quality and bandwidth on a song called "Flight of the Bumble Bees."  Sound quality deterioration was not discernable until the 4 kHz filter and below.


% trigger a \newpage just before the given reference
% number - used to balance the columns on the last page
% adjust value as needed - may need to be readjusted if
% the document is modified later
%\IEEEtriggeratref{8}
% The "triggered" command can be changed if desired:
%\IEEEtriggercmd{\enlargethispage{-5in}}

% references section

% can use a bibliography generated by BibTeX as a .bbl file
% BibTeX documentation can be easily obtained at:
% http://mirror.ctan.org/biblio/bibtex/contrib/doc/
% The IEEEtran BibTeX style support page is at:
% http://www.michaelshell.org/tex/ieeetran/bibtex/
%\bibliographystyle{IEEEtran}
% argument is your BibTeX string definitions and bibliography database(s)
%\bibliography{IEEEabrv,../bib/paper}
%
% <OR> manually copy in the resultant .bbl file
% set second argument of \begin to the number of references
% (used to reserve space for the reference number labels box)
\begin{thebibliography}{1}

\bibitem{GNU:radio}
\emph{GNU Radio}, \url{https://www.gnuradio.org/}.

\bibitem{notes:class}
Baxley, R., Henshaw, A., Nowlan, S., \& Trewhitt, E. \emph{Software-Defined Radio with GNU Radio: Theory and Application, Georgia Tech Professional Education course notes.} (2017)

\end{thebibliography}




% that's all folks
\end{document}


