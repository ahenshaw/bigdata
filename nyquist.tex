\subsection{Nyquist-Shannon Sampling Theorem}
When recording audio signals, one has to set a sampling frequency (or a frequency for recording the signal). There are several reasons why we set a sampling frequency:
\begin{enumerate}
	\item Data storage conservation,
	\item Bandwith conservation,
	\item Power conservation.
\end{enumerate}

The formula for sampling is:

\begin{equation}
x[n]=x(t)|_{t=nT_{S}}=x(nT_{S})
\end{equation}

Where \(T_{S}\) is the sampling period and \(f_{S}= \dfrac{1}{T_{S}}\) is the sampling frequency \cite{notes:class}.

However, we have to be careful not to set this sampling frequency too low, or else we run into a problem called aliasing.  Aliasing is often called undersampling, and occurs when a different time function with a lower frequency produces the same set of samples /cite{aliase:wiki}.