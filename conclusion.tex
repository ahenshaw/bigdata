\section{Conclusion}
In radio operations, frequency bands are scarce resources that are governed by national and international agencies. Every transmitter must operate within its licensed band.  Both bandwidth and power limitations are the motivation for more efficient use of bandwidth. A common way of achieving more efficient use of bandwidth is to filter frequencies occurring above the hearing range of humans (20 kHz).  While humans cannot directly hear frequencies higher than this, these frequencies can still create tones through harmonics that are noticeable to humans.  However, often times these tones are subtle and do not necessarily warrant the expense of the bandwidth they require.  Our project designed several low-pass filters to find the point that optimizes the tradeoff in sound quality and bandwidth on a song called "Flight of the Bumble Bees."  Sound quality deterioration was not discernable until the 4 kHz filter and below.
